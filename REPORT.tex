%%%% imav.tex
% This is the tex-file for the IMAV 2014 conference
% for questions / remarks / bugs regarding the files, please contact info@imavs.org
% You can use this style for your conference, as long as you refer to the IMAV 2014
% in a comment similar to this one.
% Of course, the IMAV 2014 is not liable for any aspects of its use.

\documentclass{article}
% The style file
\usepackage{imav}
% Use the postscript times font!
\usepackage{times}
\usepackage{graphicx}
\usepackage{algorithm}
\usepackage{algorithmic}
\usepackage{wasysym}
\usepackage{color}
%\numberwithin{algorithm}{chapter}
%\usepackage{algorithmicx}
% the following package is optional:
\usepackage{latexsym}

\usepackage [english]{babel}
\usepackage [autostyle, english = american]{csquotes}
\MakeOuterQuote{"}

\usepackage{acronym}


\title{Autonomous Flight of Micro Air Vehicles 2015 - Group 6}
\author{M. Bevernaegie\thanks{M.Bevernaegie@student.tudelft.nl}, C. Cheung\thanks{C.Cheung@student.tudelft.nl}, Y.I. Jenie\thanks{Y.I.Jenie-1@tudelft.nl}, S.H. Lee\thanks{S.H.Lee-2@student.tudelft.nl}, P. Lu\thanks{P.Lu-1@tudelft.nl} \\ Delft University of Technology, Delft, The Netherlands}

\begin{document}

\maketitle

\begin{abstract}
Before introducing the essential elements of successful vision-based obstacle avoidance the project and equipment are discussed. Using an AR Drone 2.0 and its HD front camera obstacles should be avoided in the TU Delft Cyber Zoo in a certain time frame without any human intervention; at the end of this quarter a competition will be done to see how well the drone will perform. The flight plan and airframe will be programmed beforehand using Paparazzi on a computer that is connected to the drone. Vision input and processing could be a demanding process for the drone and should be done efficiently. Literature on optic flow and computer vision methods is provided but should be adapted to fulfill the requirements of the project, i.e. image processing should be done wisely and iterations should be limited for the drone to react quickly. Next to the vision part, the case of flight plans with moving waypoints and fixed waypoints are discussed. Before the actual flight test was done a simulation program was developed in MATLAB for random obstacle placements and to test the vision algorithm. The actual flight in the competition showed satisfying results with no collisions and stable avoidance maneuvers but recommendations are given to optimize the avoidance system for the future.
\end{abstract}

\section{Introduction} \label{section:introduction}
This report is structured as follows. After explaining the project requirements, three main equipments for the project is introduced. Section 2 follows with the explanation of the vision algorithm used to achieve the goal of the project. Our Drone flight-plans are elaborated in section 3 by comparing two possible plans our team came up with. Two types of simulation were conducted before the tournament, elaborated in Section 4 along with the result analysis. The competition result, as well as the discussion, follows in section 5. Section 6 finally wrap the report with few concluding remarks.

\subsection{Project assignment}
The goal of the project is to develop a flight plan for an AR Drone 2.0 to fly in the Cyber Zoo with a flight time of 10 minutes (equipment will be explained in the following sections). Within this time, the drone should be avoiding obstacles in the fixed area purely based on the vision data from the front camera of the drone. Examples of the obstacles with one obvious color are given but the vision algorithm should detect an obstacle based on more possible obstacle colors. The program should be set beforehand and during the flight no human intervention is allowed, so the drone should know when to turn back into the area when flying into a corner. The goal of the flight is to cover as much distance as possible.

\subsection{AR Drone: A Quad-rotor MAV}
A Quad-rotor named AR Drone 2.0 is the chosen MAV in this project. It has a four fixed propeller that are rotated by four brushless motors, each controlled by an ATMEGA8L 8bit microcontroller. The AR.Drone is equipped with 3 cells battery that supplies 11.1 V and 1000 mAh providing power for 10 to 15 minutes of flight. The platform is very popular in MAV research field\cite{Bristeau:11}\cite{Pestana:13}\cite{Lugo:14}, since it has the capability of both hovering and fast forward cruising. In order to make the vehicle acceptable for public, a very effective control and stabilization system is mandatory for a easy piloting platform. The general view of the AR Drone 2.0, the one used for this project, is shown in Figure~\ref{f:TheDrone}. Two cameras are equipped in AR.Drone, which are not mainly intended to be used as sensor for the flight. Coupled with the IMU rotation measurement and vision algorithms, however, the camera can be used in vertical speed estimation (camera pointed down). \\

\begin{figure}
\includegraphics[width=0.5\linewidth]{Figures/TheDrone.png}
\centering
\caption{The AR Drone 2.0 from the Parrot Inc (France) - with schematic of the four propellers direction of rotation}
\label{f:TheDrone}
\end{figure}

%Quad-rotor is controlled by adjusting the thrust and rotation combination of its propellers. AR Drone propeller setup is as shown in Figure~\ref{f:TheDrone}, where two of the propeller (A and C) turn clockwise, while the other two (B and D) turn counter clockwise. On stationary hover condition, every propeller produce the same thrust from the same RPM. Torque from each motor, therefore, are canceled by the combination of propeller rotation. To have a positive pitching motion, more power is given to propeller B and C, while reducing power in A and D to keep the balance with the weight and the desired forward/backward speed. Rolling motion is controlled in similar manner, positive rolling is achieved by giving more power to A and B, while reducing power in C and D. Finally, positive yawing is controlled by giving more power to propeller A and C, while reducing power in B and D. Controlling task for stability is challenging, because the rotational motion, i.e, roll, pitch and yaw, and the translational motions, i.e., heave, sway, and surge, are coupled. The work of \cite{Bouabdallah:07} elaborated this challenge comprehensively.

%The Guidance, Navigation, and Control system of AR.Drone is supported by an a set of sensory system consisting a 3-axis accelerometer, a 2-axis gyroscope, a 1-axis gyroscope, and two ultrasonic sensors. The ultrasonic sensor is used for altitude and vertical displacement estimation, while the other is part of an Inertial Measurement Unit (IMU) that is essential for control and stability of the vehicle.  All the sensors is chosen to be as low-cost as possible, which implies that the embedded control system have to deal with a lot of bias, misalignment angles, and other errors\cite{Bristeau:11}. 


\subsection{CyberZoo and OptiTrack}
Initiated by the TU Delft Robotic Institute, the CyberZoo facility is built as  a 10 x 10 meters area covered by 7 meter high net. %The facility is located in the hangar of the Faculty of Aerospace Engineering, and it is the designated field in which the project MAV will be flown.
The Cyberzoo is equipped with a three dimensional optical motion tracking system based on called OptiTrack by NaturalPoint Inc\cite{Hansen:14}\cite{Guadarrama-Olvera:14}. The system consist of 24 camera, observed in Figure~\ref{f:OptiTrackCyberZoo},  to triangulate the position of retro-reflective markers, placed on the surface of a motion platform in a specific distance between each other. By defining a set of marker positions as a rigid body, the OptiTrack system can recognized the particular pattern and track the motion continuously in a screen, also shown in Figure~\ref{f:OptiTrackCyberZoo}. The OptiTrack then provide, through a UDP connection, a local positioning data that is almost similar with the data provided by a GPS. with the This OptiTrack system is known to be a low-cost system that have accuracy comparable with its more high end competition, the well established Vicon motion capture system\cite{Hansen:14}.

%\begin{figure}[h]
%\includegraphics[width=0.9\linewidth]{Figures/TheCyberZoo.png}
%\centering
%\caption{The cyber zoo inside the hangar of the Faculty of Aerospace Engineering, Delft University of technology. Currently covered by a thick canvas that serves as a background for computer vision for the robots inside}
%\label{f:TheCyberZoo}
%\end{figure}

\begin{figure}[h]
\includegraphics[width=0.9\linewidth]{Figures/OptiTrackCyberZoo.png}
\centering
\caption{Some camera positions on the roof of the cyber zoo. (inset) the OptiTrack GUI that combines data from the 24 camera to triangulate marker positions inside the cyberzoo}
\label{f:OptiTrackCyberZoo}
\end{figure}

%\begin{figure}[h]
%\includegraphics[width=0.7\linewidth]{Figures/Markers.png}
%\centering
%\caption{Retro-reflective marker is installed on the surface of AR.Drone hull. Each Hull have a specific position combination that can be recorded and tracked by the OptiTrack system}
%\label{f:Markers}
%\end{figure}

\subsection{Paparazzi: Open Source Autopilot}
The AR.Drone in this project will be equipped with the open source autopilot system Paparazzi. Paparazzi's Ground Control Station (GCS) is used to determined the flight plan of a vehicle to fly autonomously. %The flight plan is mainly set using combination of several blocks, or modules, embedded in the GCS, e.g., the take-off, set direction to a waypoint, go to a waypoint, or go make a circle. A wifi connection is established between Paparazzi and the vehicle for a two-way communications, enabling the AR Drone to sent its states back to the GCS and operator. The flight plans can be interrupted by the operator anytime during the flight.
Inside the CyberZoo, the states of position are derived from the OptiTrack system, and the local position data from each rigid body is sent through a UDP connection, and therefore the computer platform, running Paparazzi in the CyberZoo, is required to have two connection port. 
%Inside the cyber zoo, however, the states of position is not given directly by the drone, since it is flying in a GPS-denied environment. Instead, Paparazzi must be modified to receive data from the OptiTrack local positioning. This is done by using the NavNet module in Paparazzi, coupled with a view changes in the airframe and flight plan code. 




\section{Vision algorithm}
\subsection{Vision-based Navigation in the Literature}
\textcolor{red}{Here we talk about the vision algorithm used in the literature}
\subsection{Own Vision Algorithm}
\textcolor{red}{Here we talk about our own method and the reasons for the choice}

\section{Flight plan}
In order to remove the drift of the drone, it is necessary to implement an outer feedback loop that controls and corrects the position and velocity. This can be done by using the Optitrack system in Cyberzoo. In this section, the flight plan using the Optitrack system is proposed and discussed. \\

The proposed flight plan mainly uses the principle of "continuous relocation of waypoints", and simply does the two main things. First, if the drone detects an obstacle nearby, it either turns to the left or right. It performs this maneuver in two steps of $45^o$ to ensure the stable turning of the heading. Secondly, if the drone goes outside the predefined flight area, it turns around and sets the waypoint to the opposite side of the flight area. The turning maneuver is again carried out intermittently in four steps of $45^o$ turns. If the drone neither detects an obstacle nearby or goes out-of-bounds, it simply flies forward with a constant speed. This flight plan logic is illustrated in Figure \ref{fp1}. Note that this flight plan is called "moving waypoints" because the maneuvers like stopping and turning are carried out by actively and continuously relocating the waypoints for the desired flight trajectory.

\begin{figure}[H]
	\centering
	\includegraphics[width = 0.5\textwidth]{Figures/FP1.png}
	\caption{Logic used for the Flight Plan I - Moving Waypoints.}
	\label{fp1}
\end{figure}
\subsection{Flight Plan II - Fixed Waypoints}
\textcolor{red}{Peng's flight plan}
\subsection{Discussion and Choice of Flight plan}
\textcolor{red}{Comparison and discussion.}

\section{Simulation}
\label{section:simulation}
\subsection{Linear Simulation}
\label{subsec:lin_sim}
\textcolor{red}{Yazdi's method and results}

In order to grasp the concept of the computer vision, a linear simulation program is developed in MATLAB. The simulation program is also used to preliminary test the chosen strategy before the code implementation into the paparazzi system. This simulation also gives our team possibility to easily test the strategy in various setups of the field at the tournament. 

\subsubsection{Model and Setup}
\textbf{The tournament field} is modeled based on our observation of the cyberzoo and the rules stated in the first announcement of the tournament[REF]. It is a 10 x 10 meter square field, with a smaller 7 x 7 meter square as the obstacle field , symmetrically placed inside, as shown in three example in Figure\ref{f:TopViewSamples}.

\textbf{Poles}, our initial assumption of the possible obstacle, are placed inside the obstacle field. Since there are no setup information, we choose to randomize the pole initial parameters, i.e., the pole number, the pole diameter and the x-y position, as shown in the table\ref{t:RandomPole} below. The ranges of position (x,y) are set to make the whole body of a pole to always be inside the obstacle area, considering the diameter ranges. Note that the origin of the field model is on the bottom-left. A limitation is added in the randomization process, so that each pole are always separated by at least a certain distance, which is assumed to be 2 times the diameter of the drone. This last assumption actually limit the number of pole that can be put inside the obstacle field: we found out that randomly positioning more than 10 poles is very difficult. The height of the pole is assumed  to be constant, i.e., 2 meters high. 

\begin{table}
\caption{Range of pole parameters randomization}
\label{t:RandomPole}
\begin{tabular}{lcc}
\hline \hline
Parameter & Range & Unit \\
Pole number & 4 - 10 & - \\
Diameter & 40 - 60 & cm \\
Positions (x,y) & 2.1 - 7.9 $^*$ & m \\
\end{tabular}
\end{table}

\textbf{The Drone} is modeled from the observation of the AR-drone quad-rotor. Four circles with a heading indicator is used to visualized the drone with diameter of 0.4 m, in a top-down view as shown in Figure~\ref{f:TopViewSamples}. The drone motion is modeled using only linear kinematic equations with velocity changes, treating the drone as a point mass. Heading orientation is added to support the camera-view model, explained in the next paragraph. The initial idea is to make a complete non-linear mathematical model of the drone using simulink, and hence have a nonlinear simulation system. However, this is proven to be very difficult and time consuming, especially in determining the PID gain for its stability, and therefore was dropped. The focus of this simulation program then is only to test the strategy in various obstacle setups. The dynamic non-linear simulation will be conducted using paparazi simulator instead. 

\textbf{Camera-view} is modeled based on the pin-hole camera model, coupled with the specification of the AR-drone front camera. The result can be observed, from a random position of obstacle, in Figure~\ref{f:CameraViewSamples}. Further calibration is conducted to match the result with the sample pictures (Figure~\ref{f:Calibrations}) taken from zero altitude with a pole positioned 0.5, 1 and 2 meters infront of the camera. The calibrations results in a field of view (rounded) of 44 degree vertically, and 77 degree horizontally, as also shown by the dashed line coming from the front of the drone, in the top-down view Figure~\ref{f:TopViewSamples}. The poles are modeled as straight rectangles with specific width dictated by the randomization of poles, with a constant 2 meters of height. The color of the pole is not considered in the simulation, assuming a perfect detection of pole width. The floor is illustrated using straight lines, from the camera bottom field until the horizon at the bottom edge of the tournament field. The lines are always in the direction of the drone and only served as an illustration.

\begin{figure}
\includegraphics[width=0.9\linewidth]{Figures/TopViewSamples_3.png}
\centering
\caption{TopViewSamples 3}
\label{f:TopViewSamples}
\end{figure} 

\begin{figure}
\includegraphics[width=0.9\linewidth]{Figures/Calibrations.png}
\centering
\caption{Calibrations}
\label{f:Calibrations}
\end{figure} 

\begin{figure}
\includegraphics[width=0.9\linewidth]{Figures/CameraViewSamples_3.png}
\centering
\caption{CameraViewSamples 3}
\label{f:CameraViewSamples}
\end{figure} 

\subsubsection{Strategy Implementation Result}
The implemented strategy is as described in the previous section, summarized in the flow chart in Figure~\ref{} and Figure~\ref{}. The result is shown in series of frames (time-captured) as can be observed in Figure~\ref{}. The first strategy is tested in random obstacle configuration in the tournament field. Table~\ref{} summarized the result of number of collisions, outbounds, as well as covered distance. It should be noted that the result shown in this report is the best result for the strategy proposed, after various calibrations of speeds and threshold. 

%Figure
\begin{figure}[h]
\includegraphics[width=0.8\linewidth]{Figures/Simulation_4Poles.png}
\centering
\caption{Simulation 4Poles}
\label{f:Simulation_4Poles}
\end{figure}

\begin{figure}[h]
\includegraphics[width=0.8\linewidth]{Figures/Simulation_6Poles.png}
\centering
\caption{Simulation 6Poles}
\label{f:Simulation_6Poless}
\end{figure}

\begin{figure}[h]
\includegraphics[width=0.8\linewidth]{Figures/Simulation_8Poles.png}
\centering
\caption{Simulation 8Poles}
\label{f:Simulation_8Poles}
\end{figure}

\begin{figure}[h]
\includegraphics[width=0.8\linewidth]{Figures/Simulation_10Poles.png}
\centering
\caption{Simulation 10Poles}
\label{f:Simulation_10Poles}
\end{figure}
%Table

\subsection{MATLAB \& Paparazzi Simulation}
Another way to carry out a simulation is to use MATLAB for the simulation of the vision algorithm, and Paparazzi for the simulation of the flight plan.

\section{Competition Result and Discussion}
After tuning the control and vision threshold parameters through trial-and-error in real tests, it was confirmed that both vision algorithm and flight plan works very well with adequate sensitivity towards the distance from the orange poles and black wall. Also, it was observed that the drone could maintain a stable flight during the maneuvers, such as stopping and turning of the heading. The slow speed of the drone, together with the intermittent turning strategy, allowed the drone to stop and turn with only a marginal overshoot whenever an obstacle is detected or it goes outside the flight area. Regrettably, however, the tuning was done in the environment where the obstacles were rather spaced out widely, and in particular, the black wall was positioned near and parallel to one of the boundaries of the flight area.\\

During the actual competition, the black wall was positioned adjacently near one of the corners, and there was another orange pole next to it. What happened in the first half of the competition was: the drone moved and made a few turns, and eventually went to near the corner. After it realized that it was out of bounds, it made $180^o$ turn and attempted to come back inside. However it saw the adjacent black wall, and turned to the other side. Then it saw the orange wall and turned back to the previous heading. And it saw the black wall again, and so on. After drifting slowly towards the black wall, it touched the black wall and had to be restarted due to the low battery.\\

In the second half of the competition, the drone was started after being repositioned in the home position, so it did not go into the corner and did not get trapped again. As a result, the drone covered the traveled distance of well more than 100 m. Although this is not quite a high result compared to the other teams, our drone had the fewest collisions with the obstacle, and our team won the 6th place out of 13 teams in total.\\

In the following list, the main reasons are addressed as to why it could not be foreseen that the drone may get trapped near the boundaries/corners:
\begin{itemize}
	\item In the Paparazzi simulation, the vision algorithm was "mimicked" by a random number generator. Hence, there was never a possibility of a drone getting trapped in a place.
	\item During the real tests for tuning, the black wall was located parallel to one of the boundaries and further way from the corner. So, the drone could not be possibly trapped by the "blocking" of the wall.
\end{itemize}

To solve this problem in the future, the following strategies may be used to improve the flight plan:
\begin{itemize}
	\item "Breakthrough Maneuver": Incorporate a maneuver that when a drone makes more than two consecutive turns back and forth, turn the heading only $45^o$ and fly forward for a certain period of time while suppressing command from the vision result.
	\item "Temporary Blinding": In case the displacement of the drone is too low during a fixed time period, flexibly adjust the vision threshold parameters (e.g. the width threshold for the pole detection) such that the drone becomes less sensitive towards the obstacles nearby.
	\item "Circular Flight Area": Define the flight area as a circle, instead of a square. This will prevent the drone from going too close into a corner.
\end{itemize}

\section{Conclusion}
The drone performed well in terms of not crashing at all. The vision algorithm worked as required as all obstacles were detected and the drone turned away from them. On the other hand, it was quite slow in moving, resulting in not much distance flown. This speed had to be low to react before colliding with an obstacle. Another limitation was the dependency on the tuning. The drone had a good performace on orange and black obstacles as it has been tuned on them, but had some troubles with other colors or patterns.

The flight plan was a succes. Both the fixed waypoint and moving waypoint method could keep the drone inside the competition area and moved the drone from one waypoint to another. To prevent the drone from being trapped a breakthrough maneuver, temporary blinding and a circular flight area can give solution.



% BIBLIOGRAPHY:
% use {unsrt}:
\bibliographystyle{unsrt}
\bibliography{imav_bibliography}

% The following lines are necessary for showing the appendices correctly, do not change!
\appendix
\newcommand{\appsection}[1]{\let\oldthesection\thesection
  \renewcommand{\thesection}{Appendix \oldthesection:}
  \section{#1}\let\thesection\oldthesection}
% appendices are now indicated by appsection:

\appsection{Data}
\appsection{More data}
\end{document}

