\subsection{Own Vision Algorithm}
\label{subsec:our_vision}
In Section \ref{subsec:lit_vision}, a number of the vision-based navigation methods used in literatures were briefly introduced and discussed. Although these methods have shown considerably good results in certain work domains, our team decided to come up with more simple method for the obstacle detection and avoidance. The rationale behind this is as follows:
\begin{itemize}
	\item Lucas-Kanade optic flow generation from Harris corner detector or FAST feature detector will not work well unless the obstacles have many "features" on their surface. In the competition, monotonic orange poles and a black wall will be used, and hence we anticipate that there will not be many features available on the texture of the obstacles.
	\item For the same reason, it will be very difficult to compute the Time-to-Contact because it primarily uses the optical flow vectors.
	\item Object recognition or image classification  method can be difficult for tuning and other practical reasons. For example, in order to avoid the orange poles, one cannot simply judge from the overall color of the image as to whether the obstacle is close enough or not. This is because there are cases where multiple poles are visible in the image, although none of them are close enough to the drone. Also, the poles appearing from the "blind spot" of the camera's view are more likely to be hit than the poles located straight ahead (These are verified during the tests). In comparison to other simpler methods, the object recognition will not allow for a flexible adaption to account for all these various situations.
	\item As the drone has only one front camera, the mapping techniques, such as SLAM, will take considerably more computational time and memory. This technique does not go along well with the objective of the competition which is to enable the drone travel as much as possible while minimizing the collisions.
\end{itemize}

For these reasons, it was necessary to come up with much simpler vision algorithm which can be implemented and tested easily. The proposed vision algorithm uses a simple color thresholding method for the detection of specified obstacles, which is done in the following steps:
\begin{enumerate}
	\item Take only the bottommost row of the pixels in the image as input. Note that the floor will be visible until the distance between the drone and an obstacle standing on the floor becomes lower than a certain threshold distance, assuming that the drone moves slowly in a hovering flight at a fixed altitude. For the visualization, refer to Section \ref{section:simulation}.
	\item By setting a intensity (i.e. Y color channel)  threshold, convert the bottommost pixel row into a binary row which has the elements of value 1 and 0 (i.e. 1 if the pixel intensity value is lower than the threshold, and 0 other wise). Therefore, if the image of the black wall is captured at the bottommost row, it will be indicated by the continuous sequence of 1's.
	\item Count the number of continuous 1's, and if the sequence is longer than a certain threshold, namely the "black wall width-threshold", then mark the position as the position of a black wall. For each mark, give the weight based on the detected width. Determine either to turn left or right, depending on the weighted position of the black wall. 
	\item If there is no black wall detected (i.e. no continuous sequence of 1's exceeding the "black wall width threshold"), evaluate the Cr chroma value to detect the orange poles. Generate a binary row in the similar manner as for the black wall detection by setting an appropriate Cr threshold. 
	\item In order to react more quickly and sensitively towards the poles appearing from the sideways, evaluate the left- and right-end of the Cr binary row to see if there is any shorter continuous sequence of 1's appearing from the sides. This necessitates a different type of width-threshold on the sides, namely the "orange pole sideway threshold". If there is no detection of an orange pole on the sides, continue to the next step.
	\item Using the analogous procedure as for the black wall detection, determine either to turn left or right, depending on the weighted position of the orange pole which can be computed using another width-threshold, namely the "orange pole width-threshold".
	\item If there is still no detection of the continuous sequence of 1 which exceeds any of the above-mentioned width-thresholds, then yield a command to fly forward.
\end{enumerate}