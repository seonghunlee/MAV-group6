The performance of the AR.Drone in the competition validates both the proposed vision algorithm and the flight plan. This team drone was able to maintain a stable flight throughout the whole ten minutes without any major collision. Although two minor collision and one down time did occur, there are both due to circumstances that was not part of the algorithm design objectives. In the end our drone manage to have the fewest collision from all other team that manage to fly their drone. However, the drone only secure the 6th position out of 13 teams, due to the short covered distance. This distance result are actually confirms what the simulations before the competition suggested. However, beside the black wall trapping, we did not think that a distance more than 100 meter is actually too short in the competition that have minimum distance threshold of 40 meter, that we decide to fly the drone as slow for stability. 

Several strategies might be implemented in the future to achieve higher score in a competition, such as a Breaktrough Maneuver, Temporary Blinding, or Circular Flight Area. Adjusting the speed of the drone is also an option to have more covered distance. Overall, the competition was held well and could be a very good method to gain experience in flying an MAV autonomously, especially in preparation of the real IMAV competition

